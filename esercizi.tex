\documentclass[a4paper,10pt]{article}

\usepackage[utf8]{inputenc}
\usepackage[italian]{babel}

\usepackage{amsmath}
\usepackage{amssymb}
\usepackage{amsfonts}

\usepackage{geometry}

\geometry{verbose,a4paper,tmargin=20mm,bmargin=20mm,lmargin=20mm,rmargin=20mm}

\usepackage{graphicx}

\usepackage{epigraph}


\newcounter{esercizio}
\setcounter{esercizio}{0}
\newcommand{\Ese}{\addtocounter{esercizio}{1}\vspace{2mm}\noindent{\bf Esercizio \arabic{esercizio}. } }

\newcommand{\Pro}[3]{#1#2 - #3}

\newcommand{\Stirling}[2]
{
	\left\{\!
	\begin{array}{c}
		#1\\
		#2
	\end{array}
	\!\right\}
}

%opening
\title{Raccolta esercizi}
\author{}

\begin{document}

\maketitle

\section{Esercizi di base}

\subsection{Conteggi}
\Ese Si calcoli il numero dei possibili anagrammi per ognuna delle seguenti parole: caso, casa, satiro, teleologico, pappappero, intirizzito, disossatelo.

\Ese Si calcoli il numero degli anagrammi della parola PERDENTI per ognuna delle condizioni aggiuntive date in tabella.

\medskip
\begin{tabular}{|l|l|}
\hline
senza vocali adiacenti & \phantom{rispostaacasoodelcazzo}\\
\hline
con esattamente una coppia di vocali adiacenti& \\
\hline
con P e N non adiacenti& \\
\hline
con R e E adiacenti& \\
\hline
senza consonanti isolate & \\
\hline\end{tabular}

\medskip

\Ese Otto amici vanno al cinema. Sono rimasti 5 posti in una fila e 3 in quella davanti.
\begin{enumerate}
\item In quanti modi diversi possono sedersi?
\item In quanti modi diversi possono sedersi sapendo che Alberto e Barbara non vogliono stare vicini?
\item In quanti modi diversi possono sedersi sapendo che Francesco e Ludovico vogliono stare vicini?
\item In quanti modi diversi possono sedersi sapendo che Roberto vuol stare nella fila pi\`u vicina allo schermo?
\item In quanti modi diversi possono sedersi sapendo che Massimo vuol stare nella fila pi\`u lontana dallo schermo?
\end{enumerate}

\Ese Pescando 5 carte da un mazzo di 40 carte, qual \`e la probabilit\`a di trovarsi in mano:
\begin{enumerate}
\item una coppia?
\item un tris?
\item una doppia coppia?
\item un full?
\item un poker?
\end{enumerate}

\Ese Dire con quanti zeri termina l'espressione decimale di $2013!$

\Ese Un millepiedi ha mille piedi, mille calzini indistinguibili e mille scarpe indistinguibili. In quanti ordini diversi pu\`o infilarsi calzini e scarpe, considerando che chiaramente un piede va infilato prima in un calzino e poi in una scarpa?

\Ese Che posizione occupa la parola \textit{coccodrillo} nella lista di tutti i suoi anagrammi disposti in ordine alfabetico?

\Ese Quanti sono i numeri che, scritti in base $10$, hanno $10$ cifre e queste sono ordinate in senso (debolmente) crescente?

\Ese Quanti sono i percorsi \textit{monotoni} che partendo da $(0,0)$ arrivano a $(n,m)$? E partendo da $(0,0,0)$ arrivano a $(m,n,p)$?

\Ese Quanti sono i percorsi \textit{monotoni} che partendo da $(0,0)$ arrivano a $(10,10)$ senza passare per $(5,5)$?

\Ese Quanti sono i percorsi \textit{monotoni} che partendo da $(0,0)$ arrivano a $(n,m)$ con $n\geq m$ senza mai toccare punti $(p,q)$ con $p<q$?

\Ese In quanti modi \`e possibile scrivere $n$-coppie di parentesi (aperta e chiusa) in modo da rispettare le chiusure nel modo consueto (alcuna parentesi pu\`o essere chiusa prima che sia aperta)?

\Ese In quanti modi \`e possibile triangolare un $n$-agono convesso?

\Ese $3$ uomini e $5$ donne si siedono ad un tavolo tondo in modo che non ci siano due uomini vicini. In quanti modi possono farlo?

\Ese In quanti modi un intero $n$ pu\`o essere scritto come $a_1+\dots+a_k$ dove tutti gli $a_i$ sono interi non negativi?

\Ese In quanti modi un intero $n$ pu\`o essere scritto come somma di $3$ interi non negativi (a meno dell'ordine)?

\Ese In quanti modi si possono colorare i lati di un ottagono regolare con $8$ colori uno per lato? E usando solo bianco e nero? (due colorazioni vanno considerate indistinguibili se esiste una rotazione che manda una nell'altra)

\Ese Una pedina \`e inizialmente posta nella casella pi\`u a sinistra di uno schema $1\times 3$. Una mossa consiste nello spostare la pedina in una casella che disti al pi\`u $1$ da quella di partenza (eventualmente lasciandola ferma). Detto $s_n$ il numero dei possibili percorsi seguiti dalla pedina che impiegano $n$ mosse, mostrare che $s_{2n} = 2s_n^2+ 2s_{n-1}s_{n+1}$.

\Ese Una pulce saltella da un vertice all'altro di un tetraedro. Calcolare ricorsivamente il numero di modi con cui pu\`o tornare nel vertice di partenza dopo $n$ salti. Fare lo stesso usando un cubo.

\Ese Una stringa \`e composta di $n$ lettere. Quanti sono i possibili anagrammi di questa parola per cui ogni lettera dista al pi\`u $1$ dalla posizione che occupava in origine?

\Ese Una stringa \`e composta di $n$ lettere. Quanti sono i possibili anagrammi di questa parola per cui ogni lettera dista al pi\`u $2$ dalla posizione che occupava in origine? Scrivere una ricorsione in $n$ per il numero $a_n$ dei possibili anagrammi.

\subsection{Double Counting}
\Ese Sia $X$ un insieme di $n$ elementi e $Y=\mathcal P (X)$ l'insieme delle sue parti. Calcolare le cardinalit\`a dei seguenti insiemi:
\begin{align*}
	& \{(A,B)\in Y^2 : A\cap B = \emptyset\}\\
	& \{(A,B,C)\in Y^3 : A\cap B\cap C = \emptyset \wedge A\setminus (B\cup C)=\emptyset\}\\
	& \{(A,B)\in Y^2 : |A\cap B|<3 \wedge |(A\cup B)^c|=1\}
\end{align*}

\Ese Sia $X$ un insieme di $n$ elementi e $Y=\mathcal P (X)$ l'insieme delle sue parti. Calcolare le seguenti somme:
\begin{align*}
	\sum_{(A,B)\in Y^2} |A\cup B| \quad \quad
	& \sum_{(A,B,C)\in Y^3} |A\cap B \cap C|\\
	\sum_{(A,B,C)\in Y^3} |A\cap B \cup C^c| \quad \quad
	& \sum_{(A,B,C)\in Y^3} |A\cap B \cap C|^2
\end{align*}

\Ese Calcolare le seguenti somme:
\begin{align*}
	\sum_{k=1}^n \binom{n}{k}k^2 \quad \quad
	& \sum_{k=1}^n \binom{n}{k}k^3\\
	\sum_{k=0}^r \binom{n+k}{k} \quad \quad
	& \sum_{i=1}^n i(n+1-i)
\end{align*}

\Ese Dimostrare le seguenti identit\`a:
\begin{align*}
	\binom{n}{k} = \frac n {n-s} & \binom{n-1}{k} = \frac n s \binom{n-1}{k-1}\\
	\sum_{k=0}^n \binom{n}{k}^2 & = \binom{2n}{n}\\
	\sum_{i=0}^k \binom{n}{i}\binom{m}{k-i} & = \binom{m+n}{k}\\
	\sum_{k=0, k\text{ pari}}^n \binom{n}{k} & = \sum_{k=1, k\text{ dispari}}^n \binom{n}{k}
\end{align*}

\Ese Dimostrare che esistono solo $5$ solidi platonici.

\Ese Nella regione di Matelandia ci sono $20$ citt\`a collegate da alcune linee aeree. Ciascuno dei $18$ aerei di linea esistenti visita (periodicamente) $5$ citt\`a distinte in modo consecutivo e fissato. Sappiamo che in ogni citt\`a esistono almeno $3$ aerei che la visitano e che, per ogni coppia di citt\`a, al massimo un aereo le collega direttamente. Dimostrare che, comunque scelta una citt\`a di partenza e una di arrivo, \`e sempre possibile viaggiare dall'una all'altra usando gli aerei di Matelandia.

\Ese Ad uno stage prendono parte $72$ stagisti. Al test iniziale ciascuno risolve almeno un esercizio. Dimostrare che esiste un insieme non vuoto di problemi tale che il numero di stagisti che li ha risolti tutti \`e pari.

\subsection{Invarianti}
\Ese Si scrivono alla lavagna gli interi da $1$ a $n$ e si effettua la seguente mossa: si scelgono $2$ interi $a$ e $b$, si cancellano, e al loro posto si scrive $|a-b|$. Dire per quali $n$, alla fine si ottiene sempre un numero dispari.

\Ese Un cavallo degli scacchi si muove lungo una scacchiera $47 \times 47$. \`E possibile che riesca a percorrere la scacchiera visitando tutte le caselle una e una sola volta?

\Ese Sui vertici di un $2014$-agono regolare vi sono scritti alcuni numeri interi, all'inizio tutti nulli tranne sui vertici $A_1$ e $A_3$ dove c'\`e un $1$. Una mossa consiste nell'aumentare di $1$ due vertici adiacenti o due vertici opposti. Determinare se esiste una successione finita di mosse che porta ad avere tutti i vertici con lo stesso numero.

\Ese Sono dati i $3$ numeri $\{10,8,15\}$ e ad ogni mossa si rimpiazzano 2 numeri $a$ e $b$ tra quelli dati con $\frac{3a-4b}{5}$ e $\frac{4a+3b}{5}$. \`E possibile ottenere dopo un numero finito di mosse la terna $\{12,13,14\}$? E ottenere dei numeri $\{x,y,z\}$ con $|x-12|,|y-13|,|z-14|<1$?

\Ese In una scatola ci sono alcune palline di colore bianco, giallo e rosso; ad ogni mossa si possono scambiare $2$ palline di un colore con una di un altro colore tra quelli dati. Partendo con $7$ bianche, $9$ gialle e $15$ rosse quali configurazioni si possono raggiungere?

\Ese In ogni vertice di un $n$-agono regolare c'\`e una pedina e con una mossa se ne possono scegliere $2$ qualsiasi e spostarne una in un senso e l'altra in quello contrario. \`E possibile portarle in un unico vertice?

\Ese In ogni vertice di un $n$-agono regolare ci sono alcune pedine. Su un vertice ce n'\`e una, su quello successivo $2$, su quello ulteriore $3$ e cos\`\i via fino all'ultimo dove ce ne sono $n$. Ancora una volta con una mossa si possono scegliere $2$ qualsiasi pedine e spostarne una in un senso e l'altra in quello contrario. In quali vertici \`e possibile spostarle tutte?

\Ese Sono date $2014$ carte ciascuna con una faccia bianca e una nera. All'inizio sono disposte in fila tutte girate dal lato bianco. Con una mossa si pu\`o scegliere un blocco da $50$ con la pi\`u a sinistra girata sul bianco e capovolgerle tutte. Dimostrare che dopo un po' di mosse non si pu\`o pi\`u eseguirne alcuna.

\Ese Alberto e Barbara giocano al seguente gioco: all'inizio ci sono varie pile di monete sul tavolo e ciascuno, a turno, pu\`o effettuare una mossa a scelta tra
\begin{enumerate}
	\item togliere una moneta da una pila
	\item dividere una pila in $2$ pile ciascuna con almeno una moneta.
\end{enumerate}
Quando uno non pu\`o pi\`u fare una mossa, il gioco finisce. Dimostrare che il gioco termina.

\Ese Una scacchiera $8 \times 8$ \`e colorata in bianco e nero al solito modo. Posso applicare le seguenti trasformazioni:
\begin{itemize}
	\item scegliere una riga e invertire il colore di tutte le sue caselle,
	\item scegliere una colonna e invertire il colore di tutte le sue caselle,
	\item scegliere un quadrato $2 \times 2$ e invertire il colore di tutte le sue caselle.
\end{itemize}
Posso ottenere la situazione in cui solo la casella in alto a sinistra \`e nera?

\Ese In un pentagono (convesso) sono tracciate tutte le diagonali. Ad ogni intersezione tra lati o diagonali c'\`e una lampadina, inizialmente accesa. Con una mossa \`e possibile scegliere un segmento (sia lato che diagonale) e cambiare lo stato a tutte le lampadine che si trovano su di esso. \`E possibile spegnerle tutte?

\Ese Inizialmente \`e presente un pallino sopra il punto $(0,0)$ del piano. Se c'\`e un pallino sopra il punto $(s,t)$ \`e possibile cancellarlo da $(s,t)$ e aggiungerne uno in $(s+1,t)$ e uno in $(s,t+1)$ (se non c'\`e gi\`a in uno di questi punti). \`E possibile cancellare tutti i pallini dai punti $(x,y)$ con $x+y \leq 3$?

\subsection{Colorazioni}
\Ese \`E possibile formare un rettangolo usando i $6$ trimini del classico tetris?

\Ese Per quali interi positivi $(n,m)$ \`e possibile tassellare un rettangolo $n\times m$ con tessere $1 \times 6$? E usando tessere $1 \times k$?

\Ese Un quadrato $7 \times 7$ \`e ricoperto da $16$ tessere $1 \times 3$ non sovrapposte. In quali caselle pu\`o capitare l'unico buco $1\times 1$ rimanente?

\Ese Un quadrato $6 \times 6$ \`e tassellato con tessere $2 \times 1$. Dimostrare che si pu\`o tagliare in $2$ rettangoli senza tagliare nessuna tessera.

\subsection{Principio dell'estremale}
\Ese Nel piano sono dati $n$ punti rossi e $n$ punti verdi, a 3 a 3 non allineati. Dimostrare che esistono $n$ segmenti che non si intersecano, ciascuno dei quali congiunge $2$ punti di colore diverso.

\Ese Su tutti i punti a coordinate intere nel piano viene scritto un numero naturale in modo che ciascuno sia la media dei $4$ vicini. Dimostrare che tutti i numeri sono uguali.

\Ese Un numero dispari di cowboy, tutti che giacciono su un piano, si trovano a litigare per un bottino e tutti si sparano vicendevolmente e contemporaneamente, ciascuno puntando al pi\`u vicino (non ci sono muri od ostacoli tra loro e le loro distanze a $2$ a $2$ sono tutte diverse). Dimostrare che almeno uno sopravvive. Dimostrare che i segmenti seguiti dai proiettili non formano alcuna spezzata chiusa.

\Ese Ci sono 3 scuole, ognuna delle quali ha $n$ studenti. Ogni studente ha esattamente $n+1$ amici nelle altre due scuole complessivamente. Mostrare che si possono scegliere 3 studenti, uno per scuola, in modo che siano tutti amici.

\Ese Ad una festa ci sono diverse amicizie (almeno una), che supponiamo essere reciproche e non riflessive. Vale inoltre la seguente proprietà: se due persone hanno lo stesso numero di amici, allora non hanno amici in comune. Dimostrare che esiste una persona con esattamente un amico.

\Ese Sul piano sono disegnati alcuni punti blu e alcuni rossi, entrambi in numero finito; per ogni coppia di punti blu esiste un punto rosso sul segmento che li congiunge, e per ogni coppia di punti rossi esiste un punto blu sul segmento che li congiunge. Mostrare che tutti i punti sono allineati.

\section{Combinatoria 1}

\begin{enumerate}
	\item  Si scelgono a caso 7 interi distinti nell'insieme $\{1,2,\ldots,20\}$.

	Determinare la probabilit\`{a} che il secondo pi\`{u} piccolo sia 5.

	\item  Determinare quanti sono i divisori positivi di $$69^{5}+5\cdot 69^{4}+10\cdot 69^{3}+10\cdot 69^{2}+5\cdot 69+1.$$

	\item  Consideriamo una scacchiera $5\times 5$.

	Determinare quanti sono i cammini che partono dal quadratino in alto a sinistra, arrivano nel quadratino in basso a destra, procedono 	sempre verso il basso o verso destra, e non passano mai per il quadratino centrale.

	\item  Al termine della ``regular season'' di un campionato, le prime 7 classificate si sfidano in incontri di spareggio secondo questa regola. La n.7 incontra la n.6: la perdente ottiene il settimo premio, la vincente sfida la n.5. Dopo questo nuovo incontro, la perdente ottiene il sesto premio e la vincente sfida la n.4, e cos\`{\i} via.

	Determinare in quanti modi le sette squadre possono ricevere i sette premi.

	\item  Un laboratorio ha a disposizione una bilancia a due piatti e $n$ pesi da, rispettivamente, $1$, $3$, $9$, \ldots, $3^{n-1}$ grammi, i quali ovviamente possono essere posti indifferentemente su ogni piatto della bilancia.

	Determinare il numero di oggetti di peso diverso che si possono pesare con questa apparecchiatura.

	Determinare come cambia la risposta al punto precedente se i pesi sono potenze di due o potenze di quattro.

	\item  In un torneo, ogni giocatore incontra una ed una sola volta tutti gli altri giocatori. In ogni incontro, il vincitore guadagna 2 punti, il perdente 0 punti, mentre in caso di pareggio ogni giocatore riceve 1 punto. A torneo concluso risulta che esattamente la met\`{a} dei punti totalizzata da ciascun giocatore \`{e} stata guadagnata in partite giocate contro i 10 ultimi classificati (in particolare, ciascuno dei 10 giocatori con il punteggio pi\`{u} basso ha guadagnato met\`{a} dei suoi punti incontrandosi con gli altri 9 ultimi classificati).

	Determinare il numero di giocatori che hanno partecipato al torneo.

	\item  Sia $F$ l'insieme di tutte le $n$-uple $(A_{1},\ldots,A_{n})$, dove ogni $A_{i}$ \`{e} un sottoinsieme di $\{1,2,\ldots,1998\}$.

	Calcolare
	$$\sum_{(A_{1},\ldots,A_{n})\in F}^{}\left|A_{1}\cap\ldots\cap A_{n}\right|,
	\hspace{4em}
	\sum_{(A_{1},\ldots,A_{n})\in F}^{}\left|A_{1}\cup\ldots\cup A_{n}\right|.$$

	\item  Calcolare
	$$\sum_{k=1}^{100}\left[10\sqrt{k}\right]+\left[\frac{k^{2}}{100}\right].$$

	\item Ad uno stage partecipano $n$ ragazze $G_{1},\ldots,G_{n}$ e $2n-1$ ragazzi $B_{1},\ldots,B_{2n-1}$.  Per ogni $i=1,\ldots,n$, la ragazza $G_{i}$ conosce i ragazzi $B_{1},\ldots,B_{2i-1}$ e nessun'altro.

	Dimostrare che il numero di modi di scegliere $r$ coppie ragazzo-ragazza in modo che i componenti di ogni coppia si conoscano \`{e}
	$$\left(\!
	\begin{array}{c}
		n  \\
		r
	\end{array}\!
	\right)\frac{n!}{(n-r)!}.$$

	\item  Ad un party prendono parte $12k$ persone. Ciascuna di esse stringe la mano ad esattamente $3k+6$ persone. Si sa che esiste un numero $N$ tale che, per ogni coppia di persone $A$, $B$, il numero di invitati che stringe la mano sia ad $A$ sia a $B$ \`{e} esattamente $N$.

	Determinare per quali valori (interi) di $k$ si pu\`{o} realizzare questa situazione.

	\item $n$ partecipanti prendono parte ad una gara a punti. Finita la gara si stila la classifica in base al punteggio ottenuto da ciascuno (cosicch\'e possono esserci dei pari merito). Detto $A_n$ il numero di classifiche possibili, si scriva una ricorsione per la successione $A_n$.
	
	\item Dati interi positivi $n > r$, sia
	$$ \Stirling n r = |\{\text{partizioni di } \{1,\dots,n\} \text{ in } r \text{ sottoinsiemi non vuoti}\}|.$$
	Dimostrare che:
	\begin{align*}
		& \Stirling n 1 = \Stirling n n = 1\\
		& \Stirling{n+1}{r} = \Stirling{n}{r-1} + r \Stirling n r \\
		& \sum_{k=1}^{n} k^r = \sum_{i=1}^{r} \Stirling{r}{i} i! \binom{n+1}{i+1}
	\end{align*}
	
	\item Data una successione $a_1, \dots, a_n$ chiamiamo $r$-somma una qualsiasi somma di $r$ termini successivi. Fissati degli interi $(t,q)$, quanti termini pu\`o avere, al massimo, una successione tale che ogni $t$-somma sia negativa e ogni $q$-somma sia positiva?

\end{enumerate}
\bigskip\bigskip

\textbf{IMO Problems}: 1974/1, 1981/2, 1972/3, 1998/2, 1966/1.


\section{Combinatoria 2}


\begin{enumerate}
	\item  In una popolazione il 40\% degli individui \`{e} obeso, il 30\% mangia 1 Kg di cioccolato al giorno e di questi l'80\% sono obesi.

	Determinare la probabilit\`{a} che un obeso mangi almeno 1 Kg di cioccolato al giorno.

	\item  Determinare per quali $n$ \`{e} possibile piastrellare una scacchiera $n\times n$ usando piastrelle a forma di ``T'' composte da quattro quadratini.

	\item  Consideriamo una scacchiera $11\times 11$ privata della quarta casella della riga superiore.

	Determinare se \`{e} possibile costruire un percorso che visiti una ed una sola volta ogni casella e torni infine al punto di partenza (\`{e} possibile solo muoversi tra caselle che abbiano un lato in comune).

	\item  Il sacchetto A ed il sacchetto B contengono 3 palline nere e 5 bianche, il sacchetto C contiene 5 palline nere e 3 bianche. Si sceglie a caso un sacchetto, e poi da questo sacchetto vengono estratte (sempre a caso) due palline, reimbussolando la pallina tra la prima e la seconda estrazione.

	Sapendo che le due palline estratte sono nere, determinare la probabilit\`{a} che si sia scelto il sacchetto A.

	\item  La regione di Matelandia \`{e} costituita da un numero finito di citt\`{a}. Per garantire gli spostamenti, sono state costruite delle strade che tuttavia, per motivi di budget, sono strette e a senso unico. In ogni caso, per ogni coppia di citt\`{a}, esiste sempre un tragitto (non necessariamente diretto) che le collega in almeno uno dei due versi.

	Dimostrare che esiste almeno una citt\`{a} dalla quale si pu\`{o} raggiungere qualsiasi altra citt\`{a}, ed esiste almeno una citt\`{a} che pu\`{o} essere raggiunta partendo da ogni altra citt\`{a}.

	\item  Su un foglio di carta quadrettata si disegna un quadrato di lato 2002.  Si anneriscono poi alcuni dei lati dei quadratini interni (pensando cos\`{\i} di costruire delle pareti) in modo per\`{o} che da ogni quadratino si possa raggiungere ogni altro quadratino.

	Determinare il massimo numero di lati che possono essere anneriti.

	\item  Determinare se esiste una permutazione di $1,1,2,2,\ldots,2001,2001,2002,2002$ tale che, per ogni $k$, vi siano esattamente $k$ numeri tra le due ripetizioni di $k$.

	\item  Alberto e Barbara tracciano un grafo su di un foglio ed iniziano a giocare al seguente gioco. A turno, iniziando da Alberto, ciascun giocatore effettua una delle seguenti due mosse:
	\begin{itemize}
		\item  cancellare tre segmenti che formano un triangolo;

		\item  dati tre punti di cui due non collegati, ma collegati al terzo, cancellare i due collegamenti presenti e unire i due punti 	non collegati.
	\end{itemize}

	Il giocatore che non pu\`{o} fare mosse valide perde la partita.

	Dimostrare che l'esito della partita non dipende da come giocano Alberto e Barbara, e determinare un criterio per stabilire chi vince in funzione della configurazione iniziale.

	\item  Camillo \`{e} andato per un certo periodo a Lipsia, dove vive in un condominio con 7 tedeschi. Ogni condomino ha un posto auto riservato a lui. Camillo, da buon ita\-lia\-no, quando torna a casa parcheggia in un posto a caso; ogni tedesco invece cerca di parcheggiare al proprio posto, e solo se non ci riesce parcheggia pure lui a caso. Un giorno Camillo rientra quando tutti gli altri sono ancora fuori.

	Determinare la probabilit\`{a} che l'ultimo rientrato abbia potuto parcheggiare al proprio posto.

	\item In una tranquilla vallata vivono dodici gnomi, i cui nomi coincidono con i nomi dei mesi: Gennaio, Febbraio, \ldots, Dicembre.  Ogni gnomo abita in una casetta dipinta di azzurro o di rosa.

	Nel mese di gennaio, lo gnomo Gennaio va a trovare tutti i suoi amici.  Se nota che la maggioranza stretta dei suoi amici ha la casetta di un colore diverso dalla sua, allora entro la fine del mese egli ridipinge la sua casetta, cambiando il colore per ``adeguarsi alla maggioranza degli altri''.

	Nel mese di febbraio, tocca allo gnomo Febbraio fare visita ai suoi amici ed eventualmente ridipingere la casetta per ``adeguarsi alla maggioranza'', e cos\`{\i} via.

	Questa procedura si ripete di anno in anno.  Si dimostri che, da un certo momento in poi, nessuno gnomo avr\`{a} pi\`{u} bisogno di ridipingere la sua casetta (si supponga l'amicizia simmetrica, e che ogni gnomo non includa se stesso nella lista dei suoi amici).

	\item Un piano \`e suddiviso con $n$ rette, quante regioni distinte ci sono al massimo? E se ci sono $n$ piani per suddividere in uno spazio?

	\item $n$ interi positivi sono scritti in fila. Ad ogni passaggio, se c'\`e una coppia $(x,y)$ di interi adiacenti con $x>y$ (in quest'ordine da sinistra a destra) pu\`o essere cambiata con la coppia $(y+1,x)$ oppure $(x-1,x)$. Mostrare che sono possibili sono un numero finito di passaggi.

	\item Un grafo orientato (e finito) non contiene cicli. Sia $n$ la massima lunghezza di un cammino lungo il grafo. Dimostrare che $n$ \`e il pi\`u piccolo intero per cui si possono colorare i vertici in $n$ colori in modo che $2$ vertici dello stesso colore non siano raggiungibili l'uno a partire dall'altro.

	\item Sono date $n$ scatole contenenti ciascuna una quantit\`a non negativa di monete. Una mossa consiste nel prelevare $2$ monete da una scatola, gettarne via una e riporre l'altra in una scatola a scelta. Una configurazione di monete \`e detta \textit{felice} se con un numero finito di mosse \`e possibile fare in modo che tutte le scatole siano non vuote. Determinare tutte le configurazioni \textit{felici}.
	
	\item Determinare tutti gli interi positivi $n,m$ tali per cui \`e possibile tassellare un rettangolo $n \times m$ usando solo tessere ad ``L'' da $4$ quadretti.

	\item Un grafo ha almeno un arco. Dimostrare che \`e possibile suddividere i vertici in due sottinsiemi non vuoti $A$ e $B$ tali che il numero di archi che collegano un vertice di $A$ e un vertice di $B$ \`e maggiore di quello di archi che collegano vertici dello stesso sottoinsieme.

	\item Una circonferenza \`e divisa in $n$ parti uguali da serbatoi di carburante dove pu\`o rifornirsi un veicolo. Questo, inizialmente a secco, deve percorrere l'intero percorso lungo la circonferenza. Si sa che il carburante totale presente nei serbatoi \`e appena sufficiente per permettere al veicolo di compiere l'intero giro. Dimostrare che esiste un punto dal quale il veicolo pu\`o partire tale che riesca effettivamente a fare il giro.

	\item Un carico di $100$ lingotti (d'oro $24$ct) tutti da $1$kg sono disposti in $50$ sacchi, nessuno dei quali contiene pi\`u di $50$kg. Dimostrare che i sacchi possono essere divisi in $2$ carichi dello stesso peso.

	\item Dato $n$ intero positivo, sia $P_n=\{2^n,2^{n-1}3,\dots,3^n\}$. Dato un sottoinsieme $Y\subseteq P_n$ denotiamo con $S_Y$ la somma degli elementi di $Y$. Per ogni reale $r$ con $0\leq r \leq 3^{n+1} -	2^{n+1}$ dimostrare che esiste $Y\subseteq P_n$ tale che $0\leq r-S_Y \leq 2^n$.

\end{enumerate}
\bigskip\bigskip

\textbf{IMO Problems}: 1986/3, 1996/1.


\section{Catalogo esercizi da gare internazionali}

\subsection{Conteggi}

\Pro{IranTST}{17}{1/6} (Catalan, stime combinatoriche, livello advanced)

\subsection{Double Counting}

\Pro{IranTST}{15}{1/5} (Stima alla Cauchy-Schwarz, livello medium)

\Pro{IranTST}{14}{3/5} (Stimare il numero di configurazioni con alcune proprietà, medie aritmetiche, livello advanced)

\Pro{IranTST}{13}{3/14} (Stima alla Cauchy-Schwarz, livello medium)

\subsection{Pigeonhole}

\Pro{IranTST}{17}{2/3} (Spazi vettoriali su $\mathbb{F}_3$, livello medium/advanced)

\Pro{IranTST}{15}{1/4} (Classico problema in versione un po' più difficile, livello medium)

\Pro{IranTST}{14}{1/3} (Costruire griglie con determinate proprietà, esempio in dimensione 5, livello medium)

\Pro{IranTST}{13}{1/2} (Stimare famiglie di sottoinsiemi con determinate proprietà, induzione da medium)

\subsection{Invarianti}

\Pro{IranTST}{16}{2/3} (Gioco su griglia, casistica iniziale, argomento generale semplice per induzione, livello medium)

\subsection{Grafi}
\Pro{IranTST}{16}{1/6} (Russo tosto, difficile formulare i lemmi giusti, tante piccole sottigliezze per far funzionare
l'idea generale, livello advanced)

\Pro{IranTST}{2013}{2/10} (Integrare funzioni su grafi; livello basic)

\Pro{IranTST}{2013}{3/18} (Piastrellare con parallelogrammi, grafi sui tasselli, livello basic/medium)

\subsection{Colorazioni}

\Pro{IranTST}{13}{1/4} (Griglia esagonale infinita, schacchi in movimento, livello basic/medium)

\subsection{Costruzioni induttive}

\Pro{IranTST}{14}{2/1} (Fattorizzare $n$-cicli in trasposizioni usando alberi, induzione estesa da medium)

\subsection{Costruttivi - altro}

\Pro{IranTST}{17}{1/2} (costruire una griglia $6\times 6$ con determinate proprietà, livello basic/medium)

\Pro{IranTST}{15}{2/5} (costruire una permutazione con determinate proprietà + mostrare che in alcuni casi non esiste con tecniche
di teoria dei numeri, livello medium)

\Pro{IranTST}{14}{1/4} (Più o meno coprire il grafo completo $K_n$ con alberi lineari lunghi $n$, costruzione di un esempio grande,
livello medium/advanced)

\subsection{Geometria combinatoria}
\Pro{IranTST}{15}{3/6} (Costruire poligoni dati i lati, ovvero ordinare lunghezze in modo da rispettare sistemi lineari di
disuguaglianze, livello medium/advanced)

\Pro{IranTST}{14}{3/5} (Costruire poligono dati i vertici con determinate proprietà, livello advanced)

\Pro{IranTST}{2012}{1/3} (Rette nel piano, combinazioni lineari di polinomi che sono prodotti di rette, livello advanced)
\end{document}
